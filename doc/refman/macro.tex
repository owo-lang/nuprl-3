%Change the : character to be a normal spaced character in math mode%
\mathcode `:=58  % same as octal 003A.% 

% Definitions for the displaying of the rules.
%
\newcounter{rule}
\newcounter{sub-rule}

\newlength{\SubgoalIndent}
\settowidth{\SubgoalIndent}{\tt $H$\ }

\newlength{\ExtraIndent}
\settowidth{\ExtraIndent}{\mbox{qquad}}

\newcommand{\goal}{\noindent\tt}
\newcommand{\goalgroup}{\noindent\tt}
\newcommand{\samegoal}{\noindent\tt}

\newcommand{\continuegoal}{\hspace*{\SubgoalIndent}\hspace*{\ExtraIndent}}
\newcommand{\subgoal}{\hspace*{\SubgoalIndent}}
\newcommand{\continuesubgoal}{\hspace*{\SubgoalIndent}\hspace*{\ExtraIndent}%
\hspace*{\ExtraIndent}}
\newcommand{\goalskip}{\smallskip}

\newcommand{\mlconst}{\noindent}
\newcommand{\mlconstgroup}{\noindent}
\newcommand{\samemlconst}{\noindent}

\newcommand{\mlsection}[1]{\subsection{#1}}
\newcommand{\mlsubsection}[1]{\subsubsection*{#1}}
\newcommand{\rulesection}[1]{\subsection{#1}}
\newcommand{\rulesubsection}[1]{\subsubsection*{#1}}

\newcommand{\manentry}[1]{

\smallskip

\noindent\tid{\obeyspaces{}#1}.\ \ }

\newcommand{\uglymanentry}[1]{

\smallskip

\noindent{\tt{}#1}.\ \ }


\newcommand{\note}[1]{\typeout{#1}}

%
% either mode
%
\newcommand{\nothing}{\rule{0pt}{0pt}}
\newcommand{\tid}[1]{\hbox{\tt #1}}
\newcommand{\id}[1]{\hbox{\it #1}}
\newcommand{\nuprl}{\mbox{Nuprl}}
\newcommand{\nprl}{\nuprl}
\newcommand{\prl}{\nuprl}
\newcommand{\TRANSFORM}{\mbox{\sc transform}}
\newcommand{\CR}{\mbox{\sc return}}
\newcommand{\CMD}{\mbox{\sc command}}
\newcommand{\COPY}{\mbox{\sc copy}}
\newcommand{\ERASE}{\mbox{\sc erase}}
\newcommand{\EXIT}{\mbox{\sc exit}}
\newcommand{\DEL}{\mbox{\sc delete}}
\newcommand{\INS}{\mbox{\sc ins}}
\newcommand{\JUMP}{\mbox{\sc jump}}
\newcommand{\KILL}{\mbox{\sc kill}}
\newcommand{\LEFT}{\mbox{$\leftarrow$}}
\newcommand{\RIGHT}{\mbox{$\rightarrow$}}
\newcommand{\UP}{\mbox{$\uparrow$}}
\newcommand{\DOWN}{\mbox{$\downarrow$}}
\newcommand{\SEL}{\mbox{\sc sel}}
\newcommand{\LONG}{\mbox{\sc long}}
\newcommand{\DIAG}{\mbox{\sc diag}}
\newcommand{\MOUSE}{\mbox{\sc mouse}}
\newcommand{\PRINT}{\mbox{\sc output}}
\newcommand{\MODE}{\mbox{\sc mode}}
\newcommand{\subst}[3]{\mbox{$#1[#2/#3]$}}


%
% math-mode only
%


\newcommand{\U}[1]{\mbox{${\tt U}{#1}$}}
\newcommand{\Ui}{\U{i}}
\def\posteq#1#2{\hbox{\({\textstyle #1}\over 
  \lower 3pt\hbox{\({\textstyle #2}\)}\)}}
\newcommand{\inn}{\;in\;}
\newcommand{\turnstile}{\;\mbox{\tt >>}\;}
\newcommand{\mq}{\mbox{\ \ \ \ }}  %math-mode quad

%
%  Non math-mode only.
%
\renewcommand{\tilde}{\char'176}

%\newcommand{\CTRL}[1]{\hbox{$\uparrow\!\!\mbox{#1}$}}  %Note negative thin space.

\newcommand{\CTRL}[1]{c-#1}
\newcommand{\META}[1]{m-#1}

\newcommand{\dq}{{\verb@"@}}
\newcommand{\ml}{ML}

\newcommand{\refineusing}{{\tt refine\_using\_prl\_rule}}
\newcommand{\refine}{{\tt refine}}
\newcommand{\parsecontext}{{\tt parse\_rule\_in\_context}}
\newcommand{\autotactic}{{\tt auto\_tactic}}

\newcommand{\immediatetac}{{\tt immediate}}

\newcommand{\THENL}{{\tt THENL}}
\newcommand{\THEN}{{\tt THEN}}
\newcommand{\ORELSE}{{\tt ORELSE}}
\newcommand{\REPEAT}{{\tt REPEAT}}
\newcommand{\IDTAC}{{\tt IDTAC}}
\newcommand{\PROGRESS}{{\tt PROGRESS}}
\newcommand{\COMPLETE}{{\tt COMPLETE}}


% Maps-to arrow.
\newcommand{\map}{{$\rightarrow$}}

\newcommand{\bs}{$\backslash$}
\newcommand{\lc}{$\{$}
\newcommand{\rc}{$\}$}
\newcommand{\mysubst}[3]{#1[#3/#2]}
\newcommand{\vellipsis}{\mbox{\qquad $\left.\vdots\right.$}}
%\newcommand{\vellipsis}{$\dots$}

\newcommand {\Martinlof}{Martin-L\" of}	% dash, not hyphen!


% Stu's macros
\newcommand{\valof}{\mbox{$\leftarrow$}}
\newcommand{\ipar}[1]{\hspace*{\parindent}}
\newcommand{\ip}[1]{\hspace*{#1em}}
\newcommand{\jp}{\\[\baselineskip]}


% For tactic section.

\newcommand{\malpha}{$\alpha$}
\newcommand{\ma}{$\alpha$}
\newcommand{\mgamma}{$\gamma$}
\newcommand{\mlambda}{$\lambda$}
\newcommand{\mdelta}{$\delta$}
\newcommand{\mequiv}{{\boldmath$\equiv$}}
\newcommand{\mforall}{{\boldmath$\forall$}}
\newcommand{\mexists}{{\boldmath$\exists$}}
\newcommand{\msubset}{{\boldmath$\subset$}}
\newcommand{\mmin}{{\boldmath$\in$}}
\newcommand{\mdownarrow}{{\boldmath$\downarrow$}}
\newcommand{\muparrow}{{\boldmath$\uparrow$}}
\newcommand{\mvee}{{\boldmath$\vee$}}
\newcommand{\mneg}{{\boldmath$\neg$}}
\newcommand{\mleq}{{\boldmath$\leq$}}
\newcommand{\mrightarrow}{{\boldmath$\rightarrow$}}
\newcommand{\Tok}[1]{"#1"}
\newcommand{\MArrow}{{\boldmath$\Arrow$}}


\newenvironment{bogus}{}{}
\newenvironment{OS}{\begin{bogus}\obeyspaces}{\end{bogus}}
\newcommand{\Nu}[1]{\mbox{\tt{}#1}}
\newenvironment{Numath}{\begin{bogus}\obeyspaces\[}{\]\end{bogus}}
\newcommand{\NuColumn}[1]{\begin{array}[t]{l}#1\end{array}}
\newcommand{\Real}{\mbox{\boldmath$R$}}
\newcommand{\N}{\rule{0pt}{0pt}}

\newenvironment{RuledFigure}{\begin{figure}\hrule\vspace{1ex}}{\vspace{1ex}\hrule\end{figure}}
\newenvironment{RuledFigure*}{\begin{figure*}\hrule\vspace{1ex}}{\vspace{1ex}\hrule\end{figure*}}

% Doesn't include the figure environment.
% If 12pt or 11pt is the default, uses 8pt tt font.  If 10pt is the 
% default, tries to use the unavailable 7pt tt font.  The arguments are
% the textwidth multiplier and the typesize.
\newenvironment{Screen}[2]{
    \begin{minipage}{\textwidth}
    \centering
    \begin{minipage}{#1\textwidth}
    \singlespacing #2 \tt \obeyspaces
    }{ 
    \end{minipage}
    \end{minipage}
    }

\newcommand{\SnapshotSize}{\small}

\makeatletter 
\newcommand{\singlespacing}{\let\CS=
\@currsize\renewcommand{\baselinestretch}{1}\small\CS}
\newcommand{\singlespacingplus}{\let\CS=
\@currsize\renewcommand{\baselinestretch}{1.25}\small\CS}
\newcommand{\singlespacingplusplus}{\let\CS=
\@currsize\renewcommand{\baselinestretch}{1.35}\small\CS}
\newcommand{\doublespacing}{\let\CS=
\@currsize\renewcommand{\baselinestretch}{1.75}\small\CS}
\newcommand{\normalspacing}{\singlespacing}
\makeatother

\newcommand{\Tuple}[1]{\langle #1 \rangle}

\newcommand{\Set}[2]{\{ \, #1 \, | \: #2 \, \}}
