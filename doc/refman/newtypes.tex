\chapter{``Experimental'' Types}

\label{experimental-types}

Nuprl has been extended with two new type constructors.  The first is the
type of all canonical objects.  This type is not used in an essential way
in any existing Nuprl libraries.  It is used only to support the
``polymorphic'' style of definition described in the tactic documentation.
It is a trivial matter to incorporate this type into the standard semantic
account of Nuprl.

The second type constructor is for recursive types.  This should be
regarded as experimental since it has not been proven sound.  It is highly
likely to be sound since it is a slight variant of one which has been
rigorously verified.

\section{The ``Object'' Type}
\subsection*{formation}
\goal >> object in \U{i}  BY intro

\subsection*{canonical}
\goalgroup >> $t$ in object  BY intro [using $A$]\\*
\subgoal  [>> $t$ in $A$]


\goalskip

\goal >> $t$=$t'$ in object BY intro\\*
\subgoal  >> $t$ in object\\*
\subgoal  >> $t'$ in object
\begin{quote}\rm
where, in the second rule above, if $t$ 
is canonical then no subgoal is generated, and
if it is not canonical, then the ``using'' term must be present and the
subgoal is generated.  There are no corresponding rules for $n$-ary
equalities where $n>2$.
\end{quote}

\rm

The related ML primitives are as follows.


\manentry{object\_equality: rule}

\manentry{object\_equality\_member(t:term): rule}

\manentry{make\_object\_term: term}


\section{Simple Recursive Types}

The general form of the recursive type constructor presented in the book
can be somewhat unwieldy in practice.  A simpler form has been
implemented which does not allow mutual recursion or recursive
definition of type-valued functions.

The new terms are \tid{rec(Z.T)} and \tid{rec\_ind(r;h,z.t)}, 
for \tid{h}, \tid{z}, \tid{Z} variables and
\tid{r}, \tid{t}, \tid{T} terms.  
When closed, \tid{rec(Z.T)} is canonical and \tid{rec\_ind(r;h,z.t)} is
noncanonical.  The term \tid{rec\_ind(r;h,z.t)} is a redex with contracta
\begin{quote}
\tt t[r/z; $\backslash$z.  rec\_ind(z;h,z.t)/h].
\end{quote}
In the book there is a restriction
that r must be canonical for the term to be a redex --- the implemented
version has no such restriction, hence this \tid{rec\_ind} form has no
principal argument.  The direct computation rules do, however, treat
\tid{rec\_ind} redices as redices.

The proof rules and ML primitives follow.


\subsection*{Formation}

\begin{verbatim}
H >> rec(Z.T) in Ui by intro new Y
Y:Ui >> T[Y/Z] in Ui
\end{verbatim}
where no instance of Z bound by rec(Z.T) may occur in the domain type of
a function space, in the argument of a function application or in the
principal argument(s) of a non-application elimination form.


\subsection*{Canonical}                                            

\begin{verbatim}
H >> rec(Z.T) ext t by intro at Ui
     >> T[rec(Z.T)/Z] ext t
     >> rec(Z.T) in Ui

H >> t in rec(Z.T) by intro at Ui
    >> t in T[rec(Z.T)/Z]
    >> rec(Z.T) in Ui
\end{verbatim}

\subsection*{Noncanonical}

\begin{verbatim}
H, x:rec(Z.T), H' >> G ext g[x/y]  by unroll x new y
y:T[rec(Z.T)/Z], y=x in T[rec(Z.T)/Z] >> G[y/x] ext g
                                                                             
H, x:rec(Z.T), H' >> G ext rec\_ind(x;w,z.g[$\backslash$x.void/u]) 
                                by elim x at Ui new u,v,w,z
                     >> rec(Z.T) in Ui
u: rec(Z.T)->Ui, w: (x:{v:rec(Z.T)|u(v)}->G), z: T[{v:rec(Z.T)|u(v)}/Z] 
                     >> G[z/x] ext g 

H >> rec\_ind(r;h,z.t) in S[r/x] 
             by intro over x.S using rec(Z.T) at Ui new u,v,w
       >> r in rec(Z.T)
       >> rec(Z.T) in Ui
u: rec(Z.T)->Ui, v: w:{w:rec(Z.T)|u(w)}->S[w/x], w: T[{w:rec(Z.T)|u(w)}/Z] 
       >> t[v,w/h,z] in S[w/x]
\end{verbatim}


\subsection*{ML Primitives}

The primitives are presented as in Appendix~\ref{MLExts}

\uglymanentry{simple\_rec\_equality(y): tok -> rule}

\uglymanentry{simple\_rec\_intro(level): int -> rule}

\uglymanentry{simple\_rec\_equality\_element(level): int -> rule}

\uglymanentry{simple\_rec\_unroll\_elim(level new\_ids): int -> tok list -> rule}

\uglymanentry{simple\_rec\_elim(hyp\_num level new\_ids):  int -> int -> tok list
-> rule}

\uglymanentry{simple\_rec\_equality\_rec\_ind(level over\_id over using u v w):
   int -> tok -> term -> term -> tok list ->  rule}

\uglymanentry{destruct\_simple\_rec: term -> (tok \# term)}

\uglymanentry{make\_simple\_rec\_term: tok -> term -> term}

\uglymanentry{destruct\_simple\_rec\_ind: term -> (term \# term)}

\uglymanentry{make\_simple\_rec\_ind\_term: term -> term -> term}










