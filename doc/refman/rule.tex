% Master File: it.tex

\sloppy







% -*- Mode: Text -*-
\chapter{Refinement Rules}

\label{Rules}
	
The\index{inference rules}{}\index{proof rules}{}
\index{rules of proof}{} \nuprl{} system has been designed to accommodate the
top-down\index{top-down construction}{} construction of proofs by
refinement.
In this style one proves a judgement (i.e., a {\em goal}) by applying
a {\em refinement\index{refinement}{} rule}, thereby obtaining a set of
judgements called {\em subgoals},
and then proving each of the subgoals.
In this section we will describe the refinement rules themselves.
First we give some general comments regarding the rules and then proceed to
give a description of each rule.

\section{The Form of a Rule}
To accommodate the top-down style of the \nuprl{} system the
rules of the logic are presented in the following
{\em refinement} style.
\begin{quote}
\goal $H$ >> $T$ ext $t$ by $rule$ \\*
\subgoal $H_1$ >> $T_1$ ext $t_1$ \\*
\mbox{\qquad\qquad $\left.\vdots\right.$} \\*
\subgoal $H_k$ >> $T_k$ ext $t_k$
\end{quote}
The goal is shown at the top, and each subgoal is shown
indented underneath.
The rules are defined so that if every subgoal is true then one can show the
truth of the goal (see Section 8.1 of the book for an explanation of
truth of judgments and sequents).
If there are no subgoals ($k=0$) then the truth of the goal is axiomatic.

One of the features of the proof\index{proof editor}{} editor is that
the extraction terms are not
displayed and indeed are not immediately available.
The idea is that one can judge a term $T$ to be a type and  $T$ to be
inhabited without explicitly presenting the inhabiting object.
When one is viewing $T$ as a proposition this is convenient, as a
proposition is true if it is inhabited.
If $T$ is being viewed as a specification this allows one to
implicitly
build a program which is guaranteed to be correct for the specification.
The extraction term for a goal is built as a function of the extraction
terms of the subgoals and thus in general cannot be built until each of
the subgoals have been proved.
If one has a specific term, $t$, in mind as the inhabiting object and wants
it displayed, one can use the explicit intro rule and then show that the
type {\tt $t$ in $T$} is inhabited.
\index{inhabitation}{}
The rules have the property that each subgoal can be constructed from the
information in the rule and from the goal, exclusive of the extraction term.
As a result some of the more complicated rules require certain terms as
parameters.

Implicit in showing a judgement to be true is showing that the conclusion of
the judgement is in fact a type.
We cannot directly judge a term to be a type; rather, we show that it
inhabits a universe.
An examination of the semantic definition will reveal that this is
sufficient for our purposes.
Due to the rich type structure of the system it is not possible in general
to decide algorithmically if a given term denotes an element of a universe,
so this is something which will require proof.
The logic has been arranged so the proof that the conclusion of a goal
is a type can be conducted simultaneously with the proof that the type is
inhabited.
In many cases this causes no great overhead,
but some rules have subgoals whose only purpose is to establish that the
goal is a type, that is, that it is {\em well-formed}\index{well-formedness}{}.
These subgoals all have the form {\tt $H$ >> $T$ in \U{i}} and are referred
to as {\em well-formedness\index{well-formedness subgoals}{}} subgoals.

\section{Organization of the Rules}
The rules for reasoning about each type and objects of the type will be
presented in separate sections.
Recall from above that for each judgement of the form
\mbox{\tt $H$ >> $T$ ext $t$} where the inhabiting object $t$ is left
implicit\index{implicit term construction}{}, there is a corresponding
explicit judgement
\mbox{\tt $H$ >> $t$ in $T$ ext axiom}.
As the content of these judgements is essentially the same,
the rules for reasoning about them will be presented together.

For each type we will have the following
categories\index{rule categories}{} of rules:
\begin{itemize}
\item{\em Formation\index{formation rules}{}}\\
These rules give the conditions under which a canonical type may be judged to
inhabit a universe, thus verifying that it is indeed a type.
\item{\em Canonical\index{canonical rules}{}}\\
These rules give the conditions under which a canonical object
(implicitly or explicitly presented)
may be judged to inhabit a canonical type.
Note that the formation rules are all actually canonical rules, but it is
convenient to separate them.
\item{\em Noncanonical\index{noncanonical rules}{}}\\
These rules give the conditions under which a noncanonical object may be
judged to inhabit a type.  The elimination rules all fall in this category,
as the extract term for an elimination rule is a noncanonical term.
\item{\em Equality\index{equality rules}{}}\\
These rules give the conditions under which objects having the same outer
form may be judged to be equal.
Recall that the rules are being presented in implicit/explicit pairs,
\mbox{\tt $H$ >> $T$ ext $t$} and \mbox{\tt $H$ >> $t$ in $T$}.
The explicit judgement \mbox{\tt $H$ >> $t$ in $T$} is simply
the reflexive instance of the general equality judgement
\mbox{\tt $H$ >> $t$ = $t'$ in $T$}, and in most cases the rule for the
general form is an obvious generalization of the rule for the reflexive
form, and thus will be omitted.  
As the rules for the reflexive judgement are given in one of the other
categories, there will be no equality rules presented for some types.
\item{\em Computation\index{computation rules}{}}\\
These rules allow one to make judgements of equalities resulting from
computation.
\end{itemize}

Rules such as the {\em sequence}, {\em hypothesis} and {\em lemma} rules
which are not associated with one particular type are grouped together under
the heading ``miscellaneous''.

\section{Specifying a Rule}
In the context of a particular goal a rule is specified by giving a name
and, possibly, certain parameters.
As there are a large number of rules 
it would be unfortunate to have to remember a unique name for each one.
Instead, there are small number of generic names, and the
proof\index{proof editor}{} editor
infers the specific rule desired from the form of the goal.
In fact, for the rules dealing with specific types or objects of specific
types, there are only the names {\em intro}, {\em elim} and {\em reduce}.
The {\em intro}\index{\tt intro}{} rules are those which break down the conclusion of the goal,
and the {\em elim}\index{\tt elim}{} rules are those which use a hypothesis.
Accordingly, the first parameter of any elim rule is the declared variable
or number of the hypothesis to be used.
The {\em reduce}\index{\tt reduce}{} rules are the computation rules.
The first parameter of a reduce rule is a number that specifies which 
term of the equality is to be reduced.
Among the parameters \index{rule parameters}{}of some rules are keyword
parameters which have the following form:
\begin{itemize}
\item{\tt new\index{\tt new}{} $x_1,\cdots,x_n$}\\
This parameter is used to give new\index{new clause}{} names for hypotheses
in the subgoals.
In most cases the defaults, which are derived from subterms of the conclusion
of the goal, suffice.
For technical reasons the same variable can be declared at most once in a
hypothesis list, so if a default name is already declared a new name will
have to be given.
Whenever this parameter is used it must be the case that the names given
are all distinct and do not occur in the hypothesis list of the goal.
\item{\tt using $T$\index{\tt using}{}}, {\tt over $z.T$\index{\tt over}{}} \\
These parameters are used when judging the equality of
noncanonical\index{noncanonical terms}{} forms in
types dependent on the principal argument of the noncanonical form.
The {\tt using} parameter  specifies the type of the principal argument 
of the noncanonical form.
The value should be a canonical type which is appropriate for the
particular noncanonical form.
The {\tt over}\index{\tt over}{} parameter specifies the dependence of the
type over which the
equality is being judged on the principal argument of the form.
Each occurrence of $z$ in $T$ indicates such a dependency.
The proof editor always checks that the term obtained by substituting the
principal argument for $z$ in $T$ is
$\alpha$-convertible\index{$\alpha$-convertible}{}
to the type of the equality judgement.
\item{\tt at $\Ui$}\\
The value of this parameter is the universe level at which any type judgements
in the subgoals are to be made.  The default is $\U{1}$.
\end{itemize}

\section{Optional Parameters and Defaults}
Each rule will be presented in its most general form.
However, some of the parameters of a rule may be optional,
in which case they will be enclosed by square brackets ({\tt []}).
If a new hypothesis in a subgoal depends on an optional parameter,
and in a particular instance of the rule the optional parameter is not
given, that new hypothesis will not be added.
Such a  dependence is usually in the form of a hypothesis specifically
referring to an optional {\tt new} name.
The {\tt over} parameter discussed above is almost always optional.
If it is not given, it is assumed that the type of the equality has no
dependence on the principal argument of the noncanonical form.

The issue of default values for variable names arises when the main term 
of a goal's conclusion contains binding variables.
In general, the default values are taken to be those binding variables.
For example, the rule for explicitly showing a product to be in a universe
is
\begin{quote}
\goal $H$ >> $x$:$A$\#$B$ in \U{i} by intro [new y] \\
\subgoal >> $A$ in \U{i} \\
\subgoal $y$:$A$ >> $B$ in \U{i}
\end{quote}
The rule is presented as if a new name is given, but the default is to 
use $x$.  All the dependent types follow this general pattern.

For judging the equality of terms containing binding variables the binding
variables of the first term are in general the default values for the
``appropriate'' new hypotheses.  Consider the rule (slightly
simplified)
for showing that a
spread term is in a type:
\begin{quote}
\goal $H$ >> spread($e$;$x,y.t$) in \mysubst{$T$}{$z$}{$e$} \\
\continuegoal by intro [over $z.T$] using $w$:$A$\#$B$ [new $u,v$] \\
\subgoal $H$ >> $e$ in $w$:$A$\#$B$ \\
\subgoal $H$,$u$:$A$,$v$:\mysubst{$B$}{$w$}{$u$} >>
            \mysubst{$t$}{$x,y$}{$u,v$} in \mysubst{$T$}{$z$}{<$u$,$v$>}
\end{quote}
Here the new variables default to $x,y$.  If no new names are given and
$x$ and $y$ don't appear in $H$, then the second subgoal will be
\begin{quote}
\tt\subgoal $H$,$x$:$A$,$y$:$B$ >> $t$ in \mysubst{$T$}{$z$}{<$x$,$y$>}
\end{quote}
Again this is the general pattern for rules of this type.

\section{Hidden Assumptions}
For certain rules, we need to be able to control the free variables occuring
in the extract term.
The mechanism used to achieve this is that of {\em hidden} hypotheses.
A hypothesis is hidden when it is displayed enclosed in square brackets.
At the moment the only place where such hypotheses are added is in a subgoal
of the set\index{set rules}{} elim rule.
The intended meaning of a hypothesis being hidden is that the name of the
hypothesis cannot appear free in the extracted term; that is, that it cannot
be used computationally.
Accordingly, a hidden hypothesis cannot be the object of an {\tt elim} or 
{\tt hyp} rule.
For the rules for which the extract term is the trivial term {\tt
axiom}, the extract term contains no free variable references
and so all restrictions on the use of
hidden\index{hidden hypotheses}{} hypotheses can be removed.
The editor will remove the brackets from any hidden hypotheses
in displaying a goal of this form.

\section{Shortcuts in the Presentation}

Several conventions are used to simplify the presentation of the rules.
\begin{itemize}
\item
Almost all of the rules have the property that the list of hypotheses in a
subgoal is an extension of the hypothesis list of the goal.  For such
rules, we will show
only the new hypotheses in the subgoals.
\item Trivial extraction terms (that is, those that are just {\tt axiom}.) 
will not be exhibited.
\item
The computation rules take an integer $i$ as their first parameter.  The
conclusion must be of the form \mbox{\tt $t_1$=\ldots=$t_k$ in $T$} and $i$
is taken to refer to $t_i$ (so $1 \le i \le k$).  The rule we present only
deals with the case $i=1$ and $k=2$, but the rule will apply for any $k\ge
1$ and $i$, with the obvious changes being made.
\end{itemize}




\section{Atom}\index{atom rules}{}
\subsection*{formation}
\goal $H$ >> \U{i} ext atom by intro atom 

\goal $H$ >> atom in \U{i}  by intro

\subsection*{canonical}
\goal $H$ >> atom ext \verb+"+$\cdots$\verb+"+ by intro \verb+"+$\cdots$\verb+"+

\goal $H$ >> \verb+"+$\cdots$\verb+"+ in atom by intro
\begin{quote}\rm
where `$\cdots$' is any sequence of prl characters.


\end{quote}

\goalskip

\goal $H$ >> atom\_eq($a$;$b$;$t$;$t'$) in $T$ by intro \\*
\subgoal >> $a$ in atom \\*
\subgoal >> $b$ in atom \\*
\subgoal $a$=$b$ in atom >> $t$ in $T$ \\*
\subgoal  ($a$=$b$ in atom)->void >> $t'$ in $T$

\subsection*{computation}
\goalgroup $H$ >> atom\_eq($a$;$b$;$t$;$t'$)=$t''$ in $T$ by reduce 1 true \\*
\subgoal >> $a$=$b$ in atom \\*
\subgoal >> $t$=$t''$ in $T$

\goalskip

\goal $H$ >> atom\_eq($a$;$b$;$t$;$t'$)=$t''$ in $T$ by reduce 1 false \\*
\subgoal >> ($a$=$b$ in atom) -> void \\*
\subgoal >> $t'$=$t''$ in $T$


\section{Void}\index{void rules}{}
\subsection*{formation}

\goal $H$ >> \U{i} ext void by intro void 


\goalskip

\goal $H$ >> void in \U{i}  by intro

\subsection*{noncanonical}
\goal $H$,$z$:void >> $T$ ext any($z$) by elim $z$
                                             

\goalskip

\goal $H$ >> any($e$) in $T$  by intro \\*
\subgoal >> $e$ in void
\par

% -*- Mode: Text -*-
\section{Int}\index{integer rules}{}
\subsection*{formation}
\goal $H$ >> \U{i} ext int by intro int  


\goalskip

\goal $H$ >> int in \U{i}  by intro

\subsection*{canonical}
\goal $H$ >> int ext $c$ by intro $c$


\goalskip

\goal $H$ >> $c$ in int by intro
\begin{quote}\rm
where $c$ must be an integer constant.
\end{quote}

\subsection*{noncanonical}
\goal $H$ >> -$t$ in int \\*
\subgoal >> $t$ in int

\goalskip

\goal $H$ >> int ext $m\ op\ n$ by intro $op$ \\*
\subgoal >> int ext $m$\\*
\subgoal >> int ext $n$

\goalskip

\goal $H$ >> $m\ op\ n$ in int by intro \\*
\subgoal >> $m$ in int \\*
\subgoal >> $n$ in int
\begin{quote}\rm
where  $op$ must be one of {\tt +,-,*,/,} or {\tt mod}.
\end{quote}

\goal $H$,$x$:int,$H'$ >> $T$  
           ext ind($x$;$y,z$.$t_d$;$t_b$;$y,z$.$t_u$)
           by elim $x$ new $z$[,$y$]\\*
\subgoal $y$:int,$y$<0,$z$:\mysubst{$T$}{$x$}{$y$+1}
      >> \mysubst{$T$}{$x$}{$y$} ext $t_d$ \\*
\subgoal >> \mysubst{$T$}{$x$}{0} ext $t_b$ \\*
\subgoal $y$:int,0<$y$,$z$:\mysubst{$T$}{$x$}{$y$-1}
      >> \mysubst{$T$}{$x$}{$y$} ext $t_u$
\begin{quote}\rm
Where the optional {\tt new} name $y$ must be given if $x$ occurs free in $H'$.
\end{quote}

\goal $H$ >> ind($e$;$x$,$y$.$t_d$;$t_b$;$x$,$y$.$t_u$) in $T[e/z]$ \\*
\continuegoal by intro [over $z$.$T$] [new $u$,$v$] \\*
\subgoal  >> $e$ in int \\*
\subgoal $u$:int,$u$<0,$v$:$T[u{\tt +1}/z]$ >> $t_d[u,v/x,y]$ in $T[u/z]$ \\*
\subgoal >> $t_b$ in $T[{\tt 0}/z]$ \\*                      
\subgoal $u$:int,0<$u$,$v$:$T[u{\tt -1}/z]$ >> $t_u[u,v/x,y]$ in $T[u/z]$

\goalskip

\goal $H$ >> int\_eq($a$;$b$;$t$;$t'$) in $T$ by intro \\*
\subgoal >> $a$ in int \\*
\subgoal >> $b$ in int \\*
\subgoal $a$=$b$ in int >> $t$ in $T$ \\*
\subgoal ($a$=$b$ in int)->void >> $t'$ in $T$
 
\goalskip

\goal $H$ >> less($a$;$b$;$t$;$t'$) in $T$ by intro \\*
\subgoal >> $a$ in int \\*
\subgoal >> $b$ in int \\*
\subgoal $a$<$b$ >> $t$ in $T$ \\*
\subgoal ($a$<$b$)->void >> $t'$ in $T$

\subsection*{computation}
\goalgroup $H$ >> ind($nt$;$x$,$y$.$t_d$;$t_b$;$x$,$y$.$t_u$) = $t$ in $T$
           by reduce 1 down\\*
\subgoal >> $t_d$[$nt$,(ind($nt$+1;$x$,$y$.$t_d$;$t_b$;$x$,$y$.$t_u$))/$x$,$y$] = $t$ in $T$ \\*
\subgoal >> $nt$<0

\goalskip

\goal $H$ >> ind($zt$;$x$,$y$.$t_d$;$t_b$;$x$,$y$.$t_u$) = $t$ in $T$
          by reduce 1 base \\*
\subgoal >> $t_b$ = $t$ in $T$ \\*
\subgoal >> $zt$ = 0 in int

\goalskip

\goal $H$ >> ind($nt$;$x$,$y$.$t_d$;$t_b$;$x$,$y$.$t_u$) = $t$ in $T$
           by reduce 1 up \\*
\subgoal >> $t_u$[$nt$,(ind($nt$-1;$x$,$y$.$t_d$;$t_b$;$x$,$y$.$t_u$))/$x$,$y$] = $t$ in $T$ \\*
\subgoal >> 0<$nt$

\goalskip

\goalgroup $H$ >> int\_eq($a$;$b$;$t$;$t'$) = $t''$ in $T$ by reduce 1 true \\*
\subgoal >> $a$=$b$ in int \\*
\subgoal >> $t$=$t''$ in $T$

\goalskip

\goal $H$ >> int\_eq($a$;$b$;$t$;$t'$) = $t''$ in $T$ by reduce 1 false \\*
\subgoal >> ($a$=$b$ in int) -> void \\*
\subgoal >> $t'$ = $t''$ in $T$

\goalskip

\goalgroup $H$ >> less($a$;$b$;$t$;$t'$) = $t''$ in $T$ by reduce 1 true \\*
\subgoal >> $a$<$b$  \\*
\subgoal >> $t$=$t''$ in $T$

\goalskip

\goal $H$ >> less($a$;$b$;$t$;$t'$) = $t''$ in $T$ by reduce 1 false \\*
\subgoal >> $a$<$b$ -> void \\*
\subgoal >> $t'$=$t''$ in $T$

\par


\section{Less}\index{less rules}{}
\subsection*{formation}
\goal $H$ >> \U{i} ext $a$<$b$ by intro less \\*
\subgoal $H$ >> int ext $a$\\*
\subgoal $H$ >> int ext $b$


\goalskip

\goal $H$ >> $a$<$b$ in \U{i} by intro \\*
\subgoal $H$ >> $a$ in int \\*
\subgoal $H$ >> $b$ in int

\subsection*{equality}
\goal $H$ >> axiom in $a$<$b$ \\*
\subgoal $H$ >> $a$<$b$
\par

\section{List}\index{list rules}{}
\subsection*{formation}
\goal $H$ >> \U{i} ext $A$ list by intro list \\*
\subgoal >> \U{i} ext $A$


\goalskip

\goal $H$ >> $A$ list in \U{i} by intro \\*
\subgoal >> $A$ in \U{i}

\subsection*{canonical}
\goal $H$ >> $A$ list ext nil by intro nil at \U{i} \\*
\subgoal  >> $A$ in \U{i}


\goalskip

\goal $H$ >> nil in $A$ list by intro at \U{i} \\*
\subgoal  >> $A$ in \U{i}

\goalskip

\goal $H$ >> $A$ list ext $h$.$t$ by intro . \\*
\subgoal >> $A$ ext $h$\\*
\subgoal >> $A$ list ext $t$


\goalskip

\goal $H$ >> $a.b$ in $A$ list by intro \\*
\subgoal >> $a$ in $A$ \\*
\subgoal >> $b$ in $A$ list

\subsection*{noncanonical}
\goal $H$,$x$:$A$ list,$H'$ >> T 
           ext list\_ind($x$;$t_b$;$u,v,w$.$t_u$) \\*
\continuegoal by elim $x$ new $w$,$u$[,$v$] \\*
\subgoal >> \mysubst{$T$}{$x$}{nil} ext $t_b$\\*
\subgoal $u$:$A$,$v$:$A$ list,$w$:\mysubst{$T$}{$x$}{$v$}
          >>  \mysubst{$T$}{$x$}{$u.v$} ext $t_u$

\goalskip

\goal $H$ >> list\_ind($e$;$t_b$;$x,y,z.t_u$) in \mysubst{$T$}{$z$}{$e$} \\*
\continuegoal by intro [over $z.T$] using $A$ list [new u,v,w] \\*
\subgoal >> $e$ in $A$ list \\*
\subgoal >> $t_b$ in \mysubst{$T$}{$z$}{nil} \\*
\subgoal$u$:$A$,$v$:$A$ list,$w$:\mysubst{$T$}{$z$}{$v$} \\*
\continuesubgoal >> \mysubst{$t_u$}{$x,y,z$}{$u,v,w$} in 
                    \mysubst{$T$}{$z$}{$u$.$v$}

\subsection*{computation}
\goalgroup $H$ >> list\_ind(nil;$t_b$;$u,v,w.t_u$) = $t$ in $T$  by reduce 1 \\*
\subgoal >> $t_b$ = $t$ in $T$

\goalskip

\goal $H$ >> list\_ind($a.b$;$t_b$;$u,v,w.t_u$) = $t$ in $T$ by reduce 1 \\*
\subgoal >> \mysubst{$t_u$}{$u,v,w$}{$a,b,$list\_ind($b$;$t_b$;$u,v,w.t_u$)}
                = $t$ in $T$
\par


  
\section{Union}\index{union rules}{}
\subsection*{formation}
\goal $H$ >> \U{i} ext $A$|$B$ by intro union \\*
\subgoal >> \U{i} ext $A$ \\*
\subgoal >> \U{i} ext $B$


\goalskip

\goal $H$ >> $A$|$B$ in \U{i} by intro\\*
\subgoal >> $A$ in \U{i} \\*
\subgoal >> $B$ in \U{i}

\subsection*{canonical}
\goal $H$ >> $A$|$B$ ext inl($a$) by intro at \U{i} left \\*
\subgoal >> $A$ ext $a$\\*
\subgoal >> $B$ in \U{i}


\goalskip

\goal $H$ >> inl($a$) in $A$|$B$ by intro at \U{i} \\*
\subgoal >> $a$ in $A$ \\*
\subgoal >> $B$ in \U{i}

\goalskip

\goal $H$ >> $A$|$B$ ext inr($b$) by intro at \U{i} right \\*
\subgoal >> $B$ ext $b$\\*
\subgoal >> $A$ in \U{i}


\goalskip

\goal $H$ >> inr($b$) in $A$|$B$ by intro at \U{i} \\*
\subgoal >> $b$ in $B$ \\*
\subgoal >> $A$ in \U{i}

\subsection*{noncanonical}
\goal $H$,$z$:$A$|$B$,$H'$ >> $T$ 
          ext decide($z$;$x$.$t_l$;$y$.$t_r$)
	  by elim $z$ [new $x$,$y$] \\*
\subgoal $x$:$A$,$z$=inl($x$) in $A$|$B$ >>
         \mysubst{$T$}{$z$}{inl($x$)} ext $t_l$\\*
\subgoal $y$:$B$,$z$=inr($y$) in $A$|$B$ >>
         \mysubst{$T$}{$z$}{inr($y$)} ext $t_r$
\goalskip

\goal $H$ >> decide($e$;$x$.$t_l$;$y$.$t_r$) in \mysubst{$T$}{$z$}{$e$} \\*
\continuegoal by intro [over $z$.$T$] using $A$|$B$ [new u,v] \\*
\subgoal  >> $e$ in $A$|$B$ \\*
\subgoal $u$:$A$, $e$=inl($u$) in $A$|$B$ >>
         \mysubst{$t_l$}{$x$}{$u$} in \mysubst{$T$}{$z$}{inl($u$)} \\*
\subgoal $v$:$B$, $e$=inr($v$) in $A$|$B$ >>
         \mysubst{$t_r$}{$y$}{$v$} in \mysubst{$T$}{$z$}{inr($v$)}

\subsection*{computation}
\goalgroup $H$ >> decide(inl($a$);$x$.$t_l$;$y$.$t_r$) = $t$ in $T$ by reduce 1 \\*
\subgoal  >> \mysubst{$t_l$}{$x$}{$a$} = $t$ in $T$

\goalskip

\goal $H$ >> decide(inr($b$);$x$.$t_l$;$y$.$t_r$) = $t$ in $T$ by reduce 1 \\*
\subgoal  >> \mysubst{$t_r$}{$y$}{$b$} = $t$ in $T$
\par

\section{Function}\index{function rules}{}
\subsection*{formation}
\goal $H$ >> \U{i} ext $x$:$A$->$B$ by intro function $A$ new $x$ \\*
\subgoal >> $A$ in \U{i} \\*
\subgoal $x$:$A$ >> \U{i} ext $B$


\goalskip

\goal $H$ >> $x$:$A$->$B$ in \U{i} by intro [new $y$] \\*
\subgoal >> $A$ in \U{i} \\*
\subgoal $y$:$A$ >> \mysubst{$B$}{$x$}{$y$} in \U{i}

\goalskip

\goal $H$ >> \U{i} ext $A$->$B$ by intro function \\*
\subgoal >> \U{i} ext $A$ \\*
\subgoal >> \U{i} ext $B$


\goalskip

\goal $H$ >> $A$->$B$ in \U{i} by intro \\*
\subgoal >> $A$ in \U{i} \\*
\subgoal >> $B$ in \U{i}

\subsection*{canonical}
\goal $H$ >> $x$:$A$->$B$ ext \bs$y$.$b$ by intro at \U{i} [new $y$] \\*
\subgoal $y$:$A$ >> \mysubst{$B$}{$x$}{$y$} ext $b$ \\*
\subgoal >> $A$ in \U{i}


\goalskip

\goal $H$ >> \bs $x$.$b$ in $y$:$A$->$B$ by intro at \U{i} [new $z$] \\*
\subgoal $z$:$A$ >> \mysubst{$b$}{$x$}{$z$} in \mysubst{$B$}{$y$}{$z$} \\*
\subgoal >> $A$ in \U{i}

\subsection*{noncanonical}
\goal $H$,$f$:($x$:$A$->$B$),$H'$ >> $T$ ext \mysubst{$t$}{$y$}{$f$($a$)}
              by elim $f$ on $a$ [new $y$] \\*
\subgoal >> $a$ in $A$ \\*
\subgoal $y$:\mysubst{$B$}{$x$}{$a$},
             $y$=$f$($a$) in \mysubst{$B$}{$x$}{$a$} >> $T$ ext $t$


\goalskip

\goal $H$,$f$:($x$:$A$->$B$),$H'$ >> $T$ ext \mysubst{$t$}{$y$}{$f$($a$)}
             by elim $f$ [new $y$] \\*
\subgoal >> $A$ ext $a$ \\*
\subgoal $y$:$B$ >> $T$ ext $t$
\begin{quote}\rm
where the first of the two rules above is used when $x$ occurs free
in $B$, and the second when it
does not.
\end{quote}

\goal $H$ >> $f$($a$) in \mysubst{$B$}{$x$}{$a$} by intro using $x$:$A$->$B$ \\*
\subgoal >> $f$ in $x$:$A$->$B$ \\*
\subgoal >> $a$ in $A$

\subsection*{equality}\index{extensionality}{}
\goal $H$ >> $f$=$g$ in $x$:$A$->$B$ ext $t$ \\*
\continuegoal by extensionality [at \U{i}] [using $u$:$C$->$D$, $v$:$E$->$F$] [new $z$] \\*
\subgoal   $z$:$A$ >> $f$($z$)=$g$($z$) in $B$[$z$/$x$] ext $t$ \\*
\subgoal       >> $A$ in \U{i}\\* 
\subgoal       >> $f$ in $u$:$C$->$D$\\*
\subgoal       >> $g$ in $v$:$E$->$F$
\begin{quote}\rm
where, if the ``using'' terms are not supplied, then {\tt $x$:$A$->$B$} is
used.  This rule also applies to unary equalities.
\end{quote}




\subsection*{computation}
\goal $H$ >> (\bs $x$.$b$)($a$) = $t$ in $B$ by reduce 1 \\*
\subgoal >> \mysubst{$b$}{$x$}{$a$} = $t$ in $B$

\par


 
\section{Product}\index{product rules}{}
\subsection*{formation}
\goal $H$ >> \U{i} ext $x$:$A$\#$B$  by intro product $A$ new $x$ \\*
\subgoal >> $A$ in \U{i} \\*
\subgoal $x$:$A$ >> \U{i} ext $B$


\goalskip

\goal $H$ >> $x$:$A$\#$B$ in \U{i} by intro [new $y$] \\*
\subgoal >> $A$ in \U{i} \\*
\subgoal $y$:$A$ >> \mysubst{$B$}{$x$}{$y$} in \U{i}

\goalskip

\goal $H$ >> \U{i} ext $A$\#$B$ by intro product \\*
\subgoal >> \U{i} ext $A$\\*
\subgoal >> \U{i} ext $B$


\goalskip

\goal $H$ >> $A$\#$B$ in \U{i} by intro \\*
\subgoal >> $A$ in \U{i} \\*
\subgoal >> $B$ in \U{i}

\subsection*{canonical}
\goal $H$ >> $x$:$A$\#$B$ ext <$a$,$b$> by intro at \U{i} $a$ [new $y$] \\*
\subgoal >> $a$ in $A$ \\*
\subgoal >> \mysubst{$B$}{$x$}{$a$} ext $b$\\*
\subgoal $y$:$A$ >> \mysubst{$B$}{$x$}{$y$} in \U{i}


\goalskip

\goal $H$ >> <$a$,$b$> in $x$:$A$\#$B$ by intro at \U{i} [new $y$] \\*
\subgoal >> $a$ in $A$ \\*
\subgoal >> $b$ in \mysubst{$B$}{$x$}{$a$} \\*
\subgoal $y$:$A$ >> \mysubst{$B$}{$x$}{$y$} in \U{i}

\goalskip

\goal $H$ >> $A$\#$B$ ext <$a$,$b$> by intro \\*
\subgoal >> $A$ ext $a$ \\*
\subgoal >> $B$ ext $b$


\goalskip

\goal $H$ >> <$a$,$b$> in $A$\#$B$ by intro \\*
\subgoal >> $a$ in $A$ \\*
\subgoal >> $b$ in $B$

\subsection*{noncanonical}
\goal $H$,$z$:($x$:$A$\#$B$),$H'$ >> $T$ 
         ext spread($z$;$u,v$.$t$)
	 by elim $z$ new $u,v$\\*
\subgoal $u$:$A$,$v$:\mysubst{$B$}{$x$}{$u$},$z$=<$u$,$v$> in $x$:$A$\#$B$
         >> \mysubst{$T$}{$z$}{<$u$,$v$>} ext $t$


\goalskip

\goal $H$ >> spread($e$;$x,y.t$) in \mysubst{$T$}{$z$}{$e$} \\*
\continuegoal by intro [over $z.T$] using  $w$:$A$\#$B$ [new $u,v$] \\*
\subgoal >> $e$ in $w$:$A$\#$B$ \\*
\subgoal $u$:$A$,$v$:\mysubst{$B$}{$w$}{$u$},$e$=<$u$,$v$> in $w$:$A$\#$B$ >>
            \mysubst{$t$}{$x,y$}{$u,v$} in \mysubst{$T$}{$z$}{<$u$,$v$>}

\subsection*{computation}
\goal $H$ >> spread(<$a$,$b$>;$x,y.t$) = $s$ in $T$ by reduce 1 \\*
\subgoal >> \mysubst{$t$}{$x,y$}{$a,b$} = $s$ in $T$

\par

\section{Quotient}\index{quotient}{}
\subsection*{formation}\index{quotient/formation}{}
\goal $H$ >> \U{i} ext ($x$,$y$):$A$//$E$ by intro quotient $A$,$E$ new $x$,$y$,$z$ \\*
\subgoal >> $A$ in \U{i} \\*
\subgoal $x$:$A$,$y$:$A$ >> $E$ in \U{i} \\*	
\subgoal $x$:$A$ >> \mysubst{$E$}{$x,y$}{$x,x$} \\*
\subgoal $x$:$A$,$y$:$A$,\mysubst{$E$}{$x,y$}{$x,y$}
          >> \mysubst{$E$}{$x,y$}{$y,x$} \\*
\subgoal $x$:$A$,$y$:$A$,$z$:$A$,\mysubst{$E$}{$x,y$}{$x,y$},\
          \mysubst{$E$}{$x,y$}{$y,z$} >> \mysubst{$E$}{$x,y$}{$x,z$}


\goalskip

\goal $H$ >> ($u,v$):$A$//$E$ in \U{i} by intro new $x$,$y$,$z$\\*
\subgoal >> $A$ in \U{i} \\*
\subgoal $x$:$A$,$y$:$A$ >> \mysubst{$E$}{$u,v$}{$x,y$} in \U{i} \\*
\subgoal $x$:$A$ >> \mysubst{$E$}{$u,v$}{$x,x$} \\*
\subgoal $x$:$A$,$y$:$A$,\mysubst{$E$}{$u,v$}{$x,y$}
          >> \mysubst{$E$}{$u,v$}{$y,x$} \\*
\subgoal $x$:$A$,$y$:$A$,$z$:$A$,\mysubst{$E$}{$u,v$}{$x,y$},\
          \mysubst{$E$}{$u,v$}{$y,z$} >> \mysubst{$E$}{$u,v$}{$x,z$}

\subsection*{canonical}\index{quotient/intro}{}
\goal $H$ >> ($x,y$):$A$//$E$ ext $a$ by intro at \U{i} \\*
\subgoal >> ($x,y$):$A$//$E$ in \U{i} \\*
\subgoal >> $A$ ext $a$


\goalskip

\goal $H$ >> $a$ in ($x,y$):$A$//$E$ by intro at \U{i} \\*
\subgoal >> ($x,y$):$A$//$E$ in \U{i} \\*
\subgoal >> $a$ in $A$ 

\subsection*{noncanonical}\index{quotient/elim}{}
\goal $H$,$u$:($x,y$):$A$//$E$,$H'$ >> $t$=$t'$ in $T$ 
         by elim $u$ at \U{i} [new $v,w$]\\*
\subgoal $v$:$A$,$w$:$A$ >> \mysubst{$E$}{$x,y$}{$v,w$} in \U{i}\\*
\subgoal >> $T$ in \U{i}\\*
\subgoal $v$:$A$,$w$:$A$,\mysubst{$E$}{$x,y$}{$v,w$} >> 
\mysubst{$t$}{$u$}{$v$} = \mysubst{$t'$}{$u$}{$w$} in
                  \mysubst{$T$}{$u$}{$v$}

\subsection*{equality}\index{quotient/equality}{}
\goal $H$ >> ($x,y$):$A$//$E$ = ($u,v$):$B$//$F$ in \U{i} by intro [new $r,s$]\\*
\subgoal >> ($x,y$):$A$//$E$ in \U{i} \\*
\subgoal >> ($u,v$):$B$//$F$ in \U{i} \\*
\subgoal >> $A$ = $B$ in \U{i} \\*
\subgoal $A$=$B$ in \U{i},$r$:$A$,$s$:$A$ >>
         \mysubst{$E$}{$x,y$}{$r,s$} -> \mysubst{$F$}{$u,v$}{$r,s$} \\*
\subgoal $A$=$B$ in \U{i},$r$:$A$,$s$:$A$ >>
         \mysubst{$F$}{$u,v$}{$r,s$} -> \mysubst{$E$}{$x,y$}{$r,s$}

\goalskip

\goal $H$ >> $t$ = $t'$ in ($x,y$):$A$//$E$ by intro at \U{i} \\*
\subgoal >> ($x,y$):$A$//$E$ in \U{i} \\*
\subgoal >> $t$ in $A$ \\*
\subgoal >> $t'$ in $A$ \\*
\subgoal >> \mysubst{$E$}{$x,y$}{$t,t'$}
\par

\section{Set}\index{set rules}{}
\subsection*{formation}
\goal $H$ >> \U{i} ext \lc$x$:$A$|$B$\rc{} by intro set $A$ new $x$ \\*
\subgoal >> $A$ in \U{i} \\*
\subgoal $x$:$A$ >> \U{i} ext $B$


\goalskip

\goal $H$ >> \lc $x$:$A$|$B$\rc{} in \U{i} by intro [new y]\\*
\subgoal >> $A$ in \U{i} \\*
\subgoal $y$:$A$ >> \mysubst {$B$}{$x$}{$y$} in \U{i}

\goalskip

\goal $H$ >> \U{i} ext \lc$A$|$B$\rc{} by intro set \\*
\subgoal >> \U{i} ext $A$\\*
\subgoal >> \U{i} ext $B$


\goalskip

\goal $H$ >> \lc $A$|$B$\rc{} in \U{i} by intro \\*
\subgoal >> $A$ in \U{i} \\*
\subgoal >> $B$ in \U{i}

\subsection*{canonical}
\goal $H$ >> \lc $x$:$A$|$B$\rc{} ext $a$ by intro at \U{i} $a$ [new $y$]\\*
\subgoal >> $a$ in $A$ \\* 
\subgoal >> \mysubst{$B$}{$x$}{$a$} ext $b$\\*
\subgoal $y$:$A$ >> \mysubst{$B$}{$x$}{$y$} in \U{i}
\begin{quote}\rm where all hidden hypothesis in $H$ become unhidden in the second
subgoal.
\end{quote}

\goal $H$ >> $a$ in \lc$x$:$A$|$B$\rc{} by intro at \U{i} [new $y$]\\*
\subgoal >> $a$ in $A$ \\*
\subgoal >> \mysubst{$B$}{$x$}{$a$}\\*
\subgoal $y$:$A$ >> \mysubst{$B$}{$x$}{$y$} in \U{i}


\goalskip

\goal $H$ >> \lc $A$|$B$\rc{} ext $a$ by intro \\*
\subgoal >> $A$ ext $a$ \\*
\subgoal >> $B$ ext $b$
\begin{quote}\rm where all hidden hypotheses in $H$ become unhidden in the second
subgoal.
\end{quote}

\goal $H$ >> $a$ in \lc $A$|$B$\rc{} by intro\\*
\subgoal >> $a$ in $A$ \\*
\subgoal >> $B$ ext $b$


\subsection*{noncanonical}
\goalgroup $H$,$u$:\lc $x$:$A$|$B$\rc,$H'$ >> $T$ ext $t$ by elim $u$ \\*
\subgoal $H$, $u$:$A$, [\mysubst{$B$}{$x$}{$u$}], $H'$ >> $T$ ext $t$

\goalskip

\goal $H$,$u$:\lc $x$:$A$|$B$\rc,$H'$ >> $T$ 
                 ext (\bs $y$.$t$)($u$) 
                 by elim $u$ [new $y$] \\*
\subgoal $H$,$u$:\lc $x$:$A$|$B$\rc,$H'$, $y$:$A$, [\mysubst{$B$}{$x$}{$y$}] >> 
                  \mysubst{$T$}{$u$}{$y$} ext $t$

\begin{quote}\rm
where which one of the two above rules 
is to applied in any instance is determined by
whether or not the variable $u$ of the goal actually appears to the user
(as opposed to being generated when extraction is done).  If it does
appear, the first case applies.  Note that the subgoals have a hidden
hypothesis.
\end{quote}

\subsection*{equality}
\goal $H$ >> \lc$x$:$A$|$B$\rc{} =  \lc$y$:$A'$|$B'$\rc{}
             in \U{i} by intro [new $z$] \\*
\subgoal >> $A$ = $A'$ in \U{i} \\*
\subgoal $z$:$A$ >> \mysubst{$B$}{$x$}{$z$} = \mysubst{$B'$}{$y$}{$z$} in \U{i} \\*

\par

\section{Equality}\index{equality rules}{}
\subsection*{formation}
\goal $H$ >> \U{i} ext $a_1$=$\cdots$=$a_n$ in $A$ by intro equality $A$ $n$ \\*
\subgoal >> $A$ in \U{i} \\*
\subgoal >> $A$ ext $a_1$ \\*
\subgoal \vellipsis \\*
\subgoal >> $A$ ext $a_n$
\begin{quote}\rm
where the default for $n$ is 1.
\end{quote}

\goal $H$ >> ($a_1$=$\cdots$=$a_n$ in $A$) in \U{i} by intro \\*
\subgoal >> $A$ in \U{i} \\*
\subgoal >> $a_1$ in $A$ \\*
\subgoal \vellipsis \\*
\subgoal >> $a_n$ in $A$

\subsection*{canonical}
\goal $H$ >> axiom in ($a$ in $A$) by intro \\*
\subgoal >> $a$ in $A$

\goalskip

\goal $H$,$x$:$T$,$H'$ >> $x$ in $T$ by intro
\begin{quote}\rm
This rule does not work when T is a set or quotient term,
since intro will invoke the equality rule for the set or quotient type,
respectively.
In any case, the {\tt equality} rule can be used.
\end{quote}


\par

\section{Universe}\index{universe rules}{}
\subsection*{canonical}
\goal $H$ >> \U{i} ext \U{j} by intro universe \U{j}


\goalskip

\goal $H$ >> \U{j} in \U{i}  by intro
\begin{quote}\rm
where $j<i$.
Note that all the formation rules are intro rules for a universe type.
\end{quote}

\subsection*{noncanonical}
\begin{quote}\rm
Currently there are no rules in the system for analyzing universes.
At some later date such rules may be added.
\end{quote}

\section{Tactic}
\label{tactic-rule}
\goal $H$ >> $C$ ext $t$ by \id{tactic-text} \\*
\subgoal $H_1$ >> $C_1$ ext $t_1$ \\*
\subgoal $\vdots$ \\*
\subgoal $H_n$ >> $C_1$ ext $t_n$ \\*
\begin{quote}\rm
The text \id{tactic-text} must be an ML expression of type \tid{tactic}
(see Section~\ref{ML} for a definition of this).
The subgoals are the result of the following procedure.  The tactic
denoted by \id{tactic-text} is applied to the degenerate object of type
\tid{proof} with hypotheses $H$, conclusion $C$, no refinement and an empty
list of children.\footnote{Also, before the tactic is applied, 
the ML variable
\tid{new\_id\_initialized} is assigned the value \tid{false}.  This
variable is referenced only by functions that generate new identifiers.}
The evaluation of the application must succeed.  The
result is validation and a proof list.  The validation is applied to the
proof list.  This evaluation of the application must succeed.  The result
is a proof tree $p$.  The sequent at the root of $p$ must be the
goal \tid{$H$>>$C$}.  The subgoals \tid{$H_i$>>$C$} are the sequents, in
left-to-right order, at the unproved leaves of $p$.  The
extraction term $t$ is the extraction computed for $p$, where the terms 
$t_i$ are assumed as extraction terms for the corresponding unproved
leaves.
\end{quote}

\section{Miscellaneous}
\subsection*{hypothesis}
\goal $H$,$x$:$A$,$H'$ >> $A'$ ext $x$ by hyp $x$
\begin{quote}\rm
where $A'$ is $\alpha$-convertible to $A$
\end{quote}

\subsection*{thinning}
\goal $H$ >> $A$ ext $t$ by thinning $i_1, \ldots, i_k$\\*
\subgoal $H'$ >> $A$ ext $t$
\begin{quote}\rm
where $i_1, \ldots, i_k$  are hypothesis numbers and where $H'$
is obtained from $H$ by removing the smallest number of hypotheses that
includes the named hypotheses and such that the subgoal sequent is closed
(that is, every free variable of the conclusion or of a hypothesis is
declared to the left in the sequent).  (The rule fails if no such sequent
exists.)
\end{quote}

\subsection*{sequence}\index{sequence rule}{}
\goal $H$ >> $T$ 
        ext (\bs$x_1$.$\cdots$(\bs$x_n$.$t$)($t_n$)$\cdots$)($t_1$)\\*
\continuegoal by seq $T_1,\dots,T_n$ [new $x_1,\dots,x_n$] \\*
\subgoal >> $T_1$ ext $t_1$\\*
\subgoal $x_1$:$T_1$ >> $T_2$ ext $t_2$ \\*
\subgoal \vellipsis \\*
\subgoal $x_1$:$T_1$,$\dots$,$x_n$:$T_n$ >> $T$ ext $t$

\subsection*{lemma}\index{lemma rule}{}
\goal $H$ >> $T$
          ext \mysubst{$t$}{$x$}{term\_of($theorem$)}
          by lemma $theorem$ [new $x$] \\*
\subgoal $x$:$C$ >> $T$ ext $t$
\begin{quote}\rm
where $C$ is the conclusion of the complete theorem $theorem$.
\end{quote}

\subsection*{def}\index{def rule}{}
\goal $H$ >> $T$ ext $t$ by def ${theorem}$ [new $x$] \\*
\subgoal $x$:term\_of(${theorem}$) = ${ext}$-${term}$ in $C$ >>
$T$ ext $t$
\begin{quote}\rm
where $C$ is the conclusion of the complete theorem, ${theorem}$,
and ${ext}$-${term}$ is the term extracted from that theorem.
\footnote{This rule introduces very strong interproof dependencies.
A proof using this rule depends not only on $C$ but also on the way
$C$ is proved.}
\end{quote}

\subsection*{explicit intro}\index{explicit intro rule}{}
\goal $H$ >> $T$ ext $t$ by explicit intro $t$ \\*
\subgoal >> $t$ in $T$

\subsection*{cumulativity}\index{cumulativity rule}{}
\goal $H$ >> $T$ in \U{i}{} by cumulativity at \U{j} \\*
\subgoal >> $T$ in \U{j}
\begin{quote}\rm
where $j<i$
\end{quote}

\subsection*{substitution}\index{substitution rule}{}
\goal $H$ >> \mysubst{$T$}{$x$}{$t$}
              ext $s$
              by subst at \U{i} $t$=$t'$ in $T'$ over $x.T$ \\*
\subgoal >> $t$ = $t'$ in $T'$ \\*
\subgoal >> \mysubst{$T$}{$x$}{$t'$} ext $s$\\*
\subgoal $x$:$T'$ >> $T$ in \Ui



\subsection*{equality}\index{equality rule}{}
\goal $H$ >> $t$ = $t'$ in $T$ by equality
\begin{quote}\rm
where the equality of $t$ and $t'$ can be deduced from assumptions that are
equalities over $T$ (or equalities over $T'$ where $T = T'$ is deducible
using only reflexivity, commutativity and transitivity)
using only reflexivity, commutativity and transitivity.
\end{quote}

\subsection*{direct computation}

\goal $H$ >>  $T$
              ext $t$
              by compute using $S$ \\*
\subgoal >> $T'$ ext $t$

\goalskip

\goal $H$, $x$:$T$, $H'$ >> $T''$ 
                            ext $t$
			    by compute hyp i using $S$ \\*
\subgoal $H$, $x$:$T'$, $H'$>> $T''$ ext $t$

\begin{quote}\rm
where $S$ is
obtained from $T$ by ``tagging'' some of its subterms and $T'$ is generated
by the system by performing some computation steps on subterms of $T$, as
indicated by the tags.  A subterm $t$ is tagged by replacing it by
{\tt[[*;$t$]]} or {\tt [[$n$;$t$]]}, where $n$ is a positive integer
constant.  The latter tag causes $t$ to be computed for $n$ steps, and the
former causes computation to proceed as far as possible.  Note that many of
the computation rules, such as the one for product, are instances of direct
computation.
\end{quote}

\goal $H$ >> $T$  ext $t$  by reverse\_direct\_computation using $S$\\*
\subgoal  >> $T'$ ext $t$

\goalskip

\goal $H$, $x$:$T$, $H'$ >> $T''$  ext $t$ 
       by reverse\_direct\_computation\_hyp $i$ $S$\\*
\subgoal  $H$, $x$:$T'$, $H'$ >> $T''$ ext $t$
\begin{quote}\rm
where $S$ is a legal tagging of $T'$ and $T$ is the
result of performing the computations indicated in $S$.
\end{quote}



\subsection*{eval}

\goal >> $T$ ext $t$ by eval $b$ $\id{thm}_1$,\ldots,$\id{thm}_n$ \\*
\subgoal  >> $T'$ ext $t$

\goalskip

\goal $H$, $x$:$T$, $H'$ >> $G$ ext $t$ 
              by eval $b$ $\id{thm}_1$,\ldots,$\id{thm}_n$ \\*
\subgoal  $H$, $x$:$T'$, $H'$ >> $G$ ext $t$

\begin{quote}\rm
These rules are efficient special cases of the direct computation
rules.  For the purpose of the description below, we define a particular
evaluation function, call it $f$, which takes a list $l$ of names of
theorems, a boolean value $b$, and a term $t$, and returns a term $t'$ that
is obtained from t by performing a particular sequence of computation
steps.  Specifically, evaluation proceeds as follows.  If $t$ is
non-canonical, evaluate the principal argument(s); if not all the evaluated
principal arguments are canonical and of the proper type, then return $t$
with principal arguments replaced by their evaluated forms; otherwise,
contract the redex and evaluate the result.  If $t$ is canonical, then
continue by evaluating the immediate subterms of $t$ that are not within
the scope of a binding variable of $t$.  If $t$ is {\tt term\_of(\id{thm})}
then return $t$ if $b$ is true and \id{thm} is in the list $l$ or if $b$ is
false and \id{thm} is not in $l$, otherwise continue by evaluating the term
that {\tt term\_of(\id{thm})} stands for.  If $t$ is a variable, then
return $t$.

In the above rules, $b$ is \id{true} or \id{false}, each $\id{thm}_i$ is
identifier, and $T'$ is the result of applying $f$ to the list
[$\id{thm}_1$,\ldots,$\id{thm}_n$], $b$, and the term $T$.

\end{quote}


\subsection*{arith}\index{arith rule}{}
\goal $H$ >> $C$ by arith at \Ui
\begin{quote}\rm
The {\tt arith} rule is used to justify conclusions which follow from
hypotheses by
a restricted form of arithmetic reasoning.  Roughly speaking, {\tt arith} knows
about the ring axioms for integer multiplication and addition, the total
order axioms of $<$, the reflexivity, symmetry and transitivity of equality, and
a limited form of substitutivity of equality.  We will describe the class
of problems decidable by {\tt arith} by giving an informal account of the
procedure which {\tt arith} uses to decide whether or not $C$ follows from $H$.

The terms that {\tt arith} understands are those denoting arithmetic relations,
namely terms of the form {\tt $s$<$t$}, {\tt $s$=$t$ in int}
or the negation of a term of this form.
As the only equalities {\tt arith} concerns itself with are those of the form
{\tt $s$=$t$ in int}, we will drop the {\tt in int} and write only
{\tt $s$=$t$} in the rest of this description.
For {\tt arith} the negation of an {\tt arith}metic relation $s \theta t$ where
$\theta$ is one of $<$ or $=$ is of the form 
{\tt ($s \theta t$)->void}, which we will write as $\neg s \theta t$.
As integer equality and less-than are decidable relations,
$s \theta t$ and $\neg \neg s \theta t$
denote the same relation and will be treated identically by {\tt arith}.

The {\tt arith} rule may be used to justify goals of the form

{\goal H >> $C_1$ | $\ldots$ | $C_m$ },

where each $C_i$ is a term denoting an arithmetic relation.  If {\tt arith}
can justify the goal it will produce subgoals requiring the user to show
that the left- and right-hand sides of each $C_i$ denote integer terms.
As a convenience {\tt arith} will attempt to prove goals in which not all
of the $C_i$ are arithmetic relations; it simply ignores such disjuncts.
If it is successful on such a goal, it will produce subgoals requiring the
user to prove that each such disjunct is a type at the level given in the
invocation of the rule.

Arith analyzes the hypotheses of the goal to find relevant assumptions.  In
particular, it will maximally decompose each hypothesis into a term of the
form {\tt $A_1$ \# \ldots \# $A_n$} ($n \ge 1$), and will use as an
assumption any of the $A_i$ which are arithmetic relations of the form
describe above.

{\tt Arith} begins by normalizing the relevant formulas of the goal according to
the following conventions:
\begin {enumerate}
  \item Rewrite each relation of the form {\tt $\neg s=t$} as the
  equivalent {\tt $s<t$|$t<s$}.  
  A conclusion $C$ follows from such an
  assumption if it follows from either $s<t$ or $t<s$; hence {\tt arith}
  tries both cases.
  Henceforth, we assume that all negations of equalities have been
  eliminated from consideration.
  \item Replace all occurrences of terms which are not addition, subtraction
  or multiplication by a new variable.
  Multiple occurrences of the same term are replaced by the same variable.
  {\tt Arith} uses only facts about addition, subtraction and multiplication,
  so it
  treats all other terms as atomic.
  At this point all terms are built from integer constants and integer
  variables using $+$, $-$ and $\ast$.
  \item Rewrite all terms as polynomials
  in canonical form.  The exact
  nature of the canonical form is irrelevant (the reader may think
  of it as the form used in high school algebra texts) but has the
  important property that any two equal terms are identical.
  Each term now has the form $p+c \theta p'+c'$, where $p$ and $p'$
  are nonconstant polynomials in
  canonical form, $c$ and $c'$ are
  constants, and $\theta$ is one of $<$, $=$ or $\geq$ ($s \geq t$ is
  equivalent to $\neg t < s$).
  \item  Replace each nonconstant polynomial $p$ by a new variable,
  with each occurrence of $p$ being replaced by the same variable.
  This amounts to treating each nonconstant polynomial as an atom.
  Now each formula is of the form $z+c \theta z'+c'$.
  {\tt Arith} now decides whether or not the conclusion follows from the hypotheses
  by using the total order axioms of $<$, the reflexivity, symmetry,
  transitivity and substitutivity of $=$, and the following so-called
  {\it trivial\index{trivial monotonicity}{} monotonicity} axioms ($c$ and $d$
  are constants).
  \begin{itemize}
    \item $x \geq y, c \geq d \Rightarrow x+c \geq y+d$
    \item $x \geq y, c \leq d \Rightarrow x-c \geq y-d$
  \end{itemize}
  These rules capture all of the acceptable forms of reasoning which may be
  applied to formulas in canonical form.
\end{enumerate}

For a detailed account of the {\tt arith} rule 
and a proof of its correctness, see
the article by Tat-hung Chan cited in the book.
\end{quote}



\subsection*{monotonicity}

\goal $H$ >> $T$ ext $t$ by monotonicity  $h_1$ {\em{}op} $h_2$\\*
\subgoal  $A$ >> $T$ ext $t$
\begin{quote}\rm

where $h_1$ and $h_2$ are numbers of hypotheses that have the form
\mbox{\em$t_1$ rel $t_2$},
(the possible values of {\em rel} are described below), and {\em{}op} is
one of {\tt +}, {\tt -}, {\tt *}.  The new hypothesis $A$ is computed from
{\em{}op} and hypotheses $h_1$ and $h_2$.


The monotonicity rule performs one instance of an inference based upon
``non-trivial'' monotonicity of integer arithmetic over {\tt +}, {\tt -}, and
{\tt *}.  The new hypothesis $A$ is computed using tables (which are given
below).  There is a separate table for each of the three possible values of
{\em{}op} (there is a table for a fourth operator (/), but the code
implementing this part of the rule mysteriously vanished).  The rows and
columns of these tables are indexed by term schemas.  The new hypothesis is
the conjunction (that is, independent product) of the formulas which are in
the intersection of that row and that column of the table for {\em{}op}
such that the row and column indices match hypotheses $h_1$ and $h_2$,
respectively.  For example, if $h_1$ and $h_2$ are {\tt $t_1$>$t_2$} and
{\tt $t_3$=$t_4$}, and {\em{}op} is {\tt +}, then the new hypothesis will
be
\begin{quote}
\tt $t_1$+$t_3$>=$t_2$+$t_4$+1  \&  $t_1$+$t_4$>=$t_2$+$t_3$+1.  
\end{quote}


The monotonicity rule does not make use of definitions, so the above
formula will actually appear in terms of Nuprl primitives.  The
correspondence between the operators that appear in the tables and what is
actually used by the monotonicity rule is as follows.
\begin{quote} \tt
\begin{tabular}{lll} 
        $a$>$b$ &   {\em is represented by} &       $b$<$a$   \\
        $a$=$b$  & &                           $a$=$b$ in int \\
        $a$>=$b$ & &                           $a$<$b$ -> void   \\
        $a$\verb`~`=$b$ & &                         ($a$=$b$ in int) -> void \\
        $a$<=$b$ & &                        $b$<$a$ -> void.
\end{tabular}
\end{quote}

The tables for addition, subtraction and
multiplication are given in Figures~\ref{add-table},~\ref{sub-table}
and~\ref{mult-table}, respectively.

\begin{figure}
\begin{verbatim}
      |    z>w             z>=w            z=w             z~=w
---------------------------------------------------------------------
x>y   | x+z>=y+w+2      x+z>=y+w+1      x+z>=y+w+1         error
      |                                 x+w>=y+z+1
---------------------------------------------------------------------
x>=y  | x+z>=y+w+1      x+z>=y+w        x+z>=y+w           error
      |                                 x+w>=y+z
---------------------------------------------------------------------
x=y   | x+z>=y+w+1      x+z>=y+w        x+z=y+w         x+z~=y+w
      | y+z>=x+w+1      y+z>=x+w        x+w=y+z         x+w~=y+z
---------------------------------------------------------------------
x~=y  |   error           error         x+z~=y+w           error
                                        x+w~=y+z
  
\end{verbatim}
\caption{\protect Monotonicity table for {\tt +}. \label{add-table}}
\end{figure}

\begin{figure}
\begin{verbatim}
      |    z>w             z>=w            z=w            z~=w
---------------------------------------------------------------------
x>y   | x-w>=y-z+2      x-w>=y-z+1      x-w>=y-z+1        error
      |                                 x-z>=y-w+1
---------------------------------------------------------------------
x>=y  | x-w>=y-z+1      x-w>=y-z        x-w>=y-z          error
      |                                 x-z>=y-w
---------------------------------------------------------------------
x=y   | x-w>=y-z+1      x-w>=y-z        x-w=y-z         x-w~=y-z
      | y-w>=x-z+1      y-w>=x-z        y-w=x-z         x-z~=y-w
---------------------------------------------------------------------
x~=y  |   error           error         x-w~=y-z          error
                                        x-z~=y-w
\end{verbatim}
\caption{\protect Monotonicity table for {\tt -}. \label{sub-table}}
\end{figure}

\begin{figure}
\begin{verbatim}
      |    y>=z            y>z             y=z             y~=z
---------------------------------------------------------------------
x>0   |   xy>=xz          xy>xz           xy=xz           xy~=xz
---------------------------------------------------------------------
x>=0  |   xy>=xz          xy>=xz          xy=xz           error
---------------------------------------------------------------------
x=0   |   xy=xz           xy=xz           xy=xz           xy=xz
      |   xy=0            xy=0            xy=0            xy=0
---------------------------------------------------------------------
x<=0  |   xy<=xz          xy<=xz          xy=xz           error
---------------------------------------------------------------------
x<0   |   xy<=xz          xy<xz           xy=xz           xy~=xz
---------------------------------------------------------------------
x~=0  |   error           xy~=xz          xy=xz           xy~=xz
\end{verbatim}
\caption{\protect Monotonicity table for {\tt *}. \label{mult-table}}
\end{figure}


\end{quote}

\subsection*{division}

\goalgroup   H >> $a$ = $b$*($a$/$b$)+($a$ mod $b$)   by division\\*
\subgoal      >> $b$\verb`~`=0\\*
\subgoal      >> $a$>=0

\goalskip

\goal  H >> $a$ = $b$*($a$/$b$)-($a$ mod $b$)   by division\\*
\subgoal      >> $b$\verb`~`=0\\*
\subgoal      >> $a$<=0
 
\goalskip

\goal  H >> 0 < $a$/$b$   by division\\*
\subgoal      >> 0<$b$ \& $a$>=$b$ | $b$<0 \& $a$<=$b$

\goalskip

\goal  H >> $a$/$b$ < 0   by division\\*
\subgoal      >> 0<$b$ \& -$a$>=$b$ | $b$<0 \& -$a$<=$b$

\goalskip

\goal  H >> $a$/$b$ = 0   by division\\*
\subgoal      >> $b$\verb`~`=0\\*
\subgoal      >> $a$=0

\goalskip

\goal  H >> $a$/$b$ = $q$ \& $a$ mod $b$ = $r$   by division\\*
\subgoal      >> $b$\verb`~`= 0\\*
\subgoal      >> $a$>=0 \& $a$=$b$*$q$+$r$ | $a$<=0 \& $a$=$b$*$q$-$r$\\*
\subgoal      >> 0<=$r$\\*
\subgoal      >> 0<$b$ \& $r$<$b$ | $b$<0 \& $r$<-$b$
\begin{quote}\rm
These rules use the notations given above for the monotonicity rule.
One note on ``user-friendliness'': the rule allows conclusions which match
the patterns modulo commutativity over * and + and = .  For example, 0 =
a/b will ``match'' the fifth division rule format; and (a/b)*b + (a mod b) =
a will match the first pattern.  Thus one need not memorize specific
details of the patterns, but only remember the ``basic format'' in order to
know when the rule can be successfully invoked.
\end{quote}



\rm
